\documentclass[a4paper,10pt]{article}
\usepackage[utf8]{inputenc}%PAQUETE PARA CORRECCION DE TIPOGRAFIA PARA ESPAÑOL
\usepackage[english,spanish]{babel}%PAQUETE DICCIONARIO ESPAÑOL
\usepackage{amssymb}%PAQUETES COMPLEMENTO PARA TIPOGRAFIA MATEMATICA
\usepackage[pdftex]{graphicx}%PAQUETE PARA LA INCLUSION DE IMAGENES jpg, png Y pdf
%\usepackage{pdfpages}
\renewcommand{\rmdefault}{phv} % Arial
\renewcommand{\sfdefault}{phv} % Arial
\author{Instituto Dan Beninson}%AUTOR DEL DOCUMENTO
\date{\today}%FECHA DE EDICION DEL DOCUMENTO
\title{Manual del usuario LectorMySQL}%TITULO DEL DOCUMENTO

\begin{document}
\maketitle
\abstract{El sistema de administración IDB es una máscara que facilita el manejo de datos en tablas en bases de datos. El sistema funciona en un servidor web y es multiusuario. Los datos se guardan en el servidor y pueden ser compartidos entre todos los usuario de acuerdo a cómo lo establece el administrador. El sistema IDB lee todas las bases de datos instaladas en el sistema y crea automáticamente formularios de alta, modificación, baja, consultas avanzadas, informes y otros. 

El próposito de este sistema es dar soporte administrativo a todas las áreas del instituto. El sistema cuenta con un núcleo principal llamado LectorMySQL y subsistemas dedicados a cada una de las necesidades. Así el Sistema IDB es de uso cotidiano para el manejo de información de alumnos, recursos, notas, biblioteca y otros.

El presente documento pretende ser un manual para el usuario. En él operador del sistema IDB podrá encontrar todaslas fnciones disponibles para obtener un total funcionamiento y gestión de sus datos.
}

\clearpage

\tableofcontents
\clearpage

\section{Introduccion}

\subsection{¿Qué es el LectorMySQL y para que sirve?}

Es una aplicación web, multiusuario que permite la gestión de tablas en una mase de datos MySQL. 
Más precisamente, es una interfaz entre el usuario y la información guardada en una base de datos.

Este sistema de gestión permite operaciones básicas de consulta y maneja un 
proceso de actualización, reorganización, alta y baja de información en forma de tablas. 

Genera automáticamente formularios para interactuar con todo tipo de tablas, optimizando el uso y unificando 
el tipo de información. Permite asimismo la gestión compartida de la información ya que el acceso es 
restringido por un usuario Administrador que es quien define los perfiles.

\subsection{¿Qué es una base de datos?}
Se llama base de datos ---o también banco de datos--- a un conjunto de información 
perteneciente a un mismo contexto, ordenada en tablas de modo sistemático para su 
recuperación, análisis y/o transmisión.

El manejo de las bases de datos se lleva mediante sistemas de gestión (llamados 
DBMS por sus siglas en inglés: Database Management Systems o Sistemas de Gestión 
de Base de Datos). MySQL es uno de ellos. Es un sistema de gestión de información que opera 
sobre un servidor web y que queda a la espera de una petición por parte del usuario.

En la conformación de una base de datos se pueden seguir diferentes modelos y 
paradigmas, cada uno dotado de características, ventajas y dificultades, 
haciendo énfasis en su estructura organizacional, su jerarquía, su capacidad de 
transmisión o de interrelación, etc. Esto se conoce como modelos de base de 
datos y permite el diseño y la implementación de algoritmos y otros mecanismos 
lógicos de gestión, según sea el caso específico.

El LectorMySQL es una máscara o interfaz que se conecta a un servidor de base de datos MySQL ---por medio de un servidor web---
 e interactúa con ellas sin importar el tipo o formato de tablas existentes.

\subsection{Tecnología}

\section{Ingreso al sistema}

(Captura de pantalla)

Para ingresar al Sistema IDB necesitamos previamente un Navegador web (Chrome, Firefox, Opera, etc.). Este sistema al tener una interfase web, el usuario ingresa a través de la dirección física o IP del Sistema IDB en la barra de dirección de su navegador preferido. Luego del direccionamiento nos encontramos con una serie de menús. El primero corresponde a “Bases de Datos Activas” en el cual podemos realizar consultas y ABM\footnote{¿Que es ABM?. si usamos la sigla hay que definirla primero.} de todas las bases de datos como por ejemplo Alumnos IDB, Biblioteca IDB, Encuestas IDB, etc.). El segundo menú corresponde a los “Sistemas Dedicados” que complementan al uso del sistema por ejemplo, Contabilidad y Recursos, Sistema de Compras, Calendario, Sistema de Inasistencias.

(Captura de pantalla)

Para el primer grupo de opciones al querer ingresar nos solicita una autenticación (Usuario y Contraseña), permitiendo o negando el acceso a las diferentes bases de datos. En el segundo grupo nos re-direcciona a un sub-menú referente al año el cual queremos gestionar y luego solicita la autenticación del usuario. 
 
\subsection{Elementos del menú principal\label{ElementosDelMenuPrincipal}}

El menú principal consta de una barra de acceso al menú de inicio con el objetivo de poder intercambiar las bases de datos a las cuales queremos acceder. 
Luego un mensaje indicándonos la base de datos activa y el usuario logueado.
Este continúa con un menú que contiene diversas opciones disponibles para el usuario: la Emisión de Certificados, Cambiar la clave de ingreso, Importar datos, Eliminación de registros y la papelera.

(Captura de pantalla)

El menú principal a su vez muestra un formulario en el cual se reflejan las búsquedas guardadas por el Usuario y los ABMC\footnote{¿Que es ABMC?. Idem anterior.} de las mismas. Por debajo con el conjunto de opciones que son las encargadas de manipular los datos de la base en la que estamos parados.
Dichas opciones contienen las distintas tablas que forman los datos y una serie de links para el ABMC de cada una de ellas.



\section{Entorno de Usuario}
\subsection{Tabla de usuario}
Cada usuario al ingresar a la base de datos se encuentra con un formato generico para la dispocision de los distintos menus. Aparte de esto el sistema muesra un conjunto de busquedas generadas por el usuario propias del mismo. Estas busquedas quedan almacenadas para el uso diario, y no son comparticas para los distinto susuarios. Por lo tanto este menu que muestra lo que en el punto siguente definimos como busquedas particulares quedan como configuraciones en la interfaz del usuario personalizada.


\subsection{Búsquedas particulares}
En el uso diario del sistema IDB y las consultas frecuentes a los mismos registros puede generar un uso tedioso al momento de realizar la búsqueda necesaria. Para ello se genero lo que se denominan búsquedas particulares, estas nos permiten tener un acceso más rápido a dichas consultas.

(Captura de pantalla)

Las búsquedas particulares son búsquedas guardadas por el usuario para acceso frecuente, al momento de realizar una búsqueda se puede pedir que ese grupo de registros lo guarde como una búsqueda particular, generando un link en el menú principal. Este link consta de un titulo formado por automáticamente por los principales registros  de los  datos que el usuario consultó. Una vez realizada la búsqueda particular se puede trabajar en estos registros si necesitamos consultarlos,  modificarlos, duplicarlos y/o exportarlos.

(Captura de pantalla)

Las búsquedas particulares también son utilizadas al momento de necesitar realizar una Emisión de Certificados como también la eliminación de registros. 
Cuando el usuario desea dejar de tener una búsqueda particular en su menú, este puede limpiar el  acceso rápido utilizando el icono correspondiente a borrar búsqueda. Esta acción no modifica ni elimina los registros con los que se trabajo, solo borra el acceso rápido a ellos.
    
\subsection{Cambio de contraseña}

El sistema IDB dependiendo de cada una de las distintas base de datos deja acceder o no al usuario, por lo cual cada base de datos contiene los usuarios y las contraseñas de estos independientemente de cada una de las distintas bases de datos.

(Captura de pantalla)

Para el cambio de contraseña el usuario tiene que acceder a la base de datos donde desea cambiar su password. Luego del acceso a dicha base de datos nos dirigimos al link “Cambiar clave de ingreso”, el sistema nos redirige a una pantalla con un formulario para completar con la contraseña actual y la nueva contraseña. Completamos  con los datos correspondientes y presionamos el botón de aceptar. 
El sistema nos informa que los datos fueron modificados correctamente. Realizado este proceso ya disponemos de una nueva contraseña con el cual podemos ingresar a la base de datos que seleccionamos.

\subsection{Importación de datos}
Cuando un usuario tiene la necesidad de realizar una carga de datos extensa el sistema contiene una función la cual nos facilita la carga de estos. Implementando un archivo CSV logramos importar los datos y plasmarlos directamente en la base de datos que queramos.
Esta opcion se encuentra en el menu principal de cada una de las Base de datos  al ingresar a esta nos indica las caracteristica sque tiene que tener el archivo CSV para importar los datos que necesitamos. Seleccionamos el archivo, e indicamos el nombre que queremos darle a nuestra tabla. una vez realizada esta operacion ya podemos verificar la carga de los datos con la nueva tabla que generamos.

Definición de archivos CSV: Un csv (comma separated values) es un archivo de texto que almacena los datos en forma de columnas, separadas por coma y las filas se distinguen por saltos de línea. Es una forma muy sencilla de representar la información. Normalmente para importar o exportar de bases de datos de unas aplicaciones.


\subsection{Compartir datos con otros usuarios}
Muchas veces necesitamos trabajar en conjunto con otros usuarios sobre el misma conjunto de datos. Para esto se usa la funcion de compartir datos, en la cual, nosotros podemos compartir los datos que necesitemos y asi nuestro colega podera ver facilmente el conjunto de datos con el cual necesitamos trabajar en equipo. Al compartir los datos los distintos usuarios van a tener acceso total a los datos y podran manipularlos conjuntamente.
Una vez realizada la tarea nosotros podemos darnos de baja de ese grupo compartido eliminando la busqueda particular pero hay que tener en cuenta que los demas usuarios seguiran teniendo acceso a dichos datos.


\section{Funciones de Tabla}
Al momento de una carga grande de registros dentro de una misma tabla podemos realizar una duplicacion de registros para plazmar nuestros nuevos datos. Teniendo la posibilidad de modificar dichos registros alterando las variables que necesitemos
Para esto nosotros podemos realizar una actualizacion de una tabla completa si es necesario. Al encontrarse dentro de una tabla, esta se podra editar completamente. Si nosotros tomamos esta decicion, visualizaremos todos los registros que pertenecen a la tabla con la posibilidad de alterar cualquier campo a nuestra voluntad (ATENCION! una vez cambiado el campo y aceptado los cambios no hay posibilidad de restaurar la tabla a una version anterior), ante cualquier eventualidad o error de carga de datos se puede repetir esta actualizacion de tabla cuantas veces lo desceemos.




\subsection{Estructura de una tabla de base de datos}

Cuando el administrador da de alta una tabla en una base dados MySQL, es necesario definir una serie de parámetros y características de cada columna o \textbf{campo}. Es decir, que toda tabla del sistema tiene definidos en detalle el tipo de datos para cada columna, para cada tabla. Por ejemplo definir si el tipo es numérico, de fecha, texto, etc. Esto es necesario para poder realizar una validación\footnote{Se denomina validación al control que se realiza sobre los datos antes de que ingresen en una tabla.} de los datos y garantizar asi la homogeneidad del contenido de la columna y por ende de la tabla.

Por defecto el sistema agrega a todas las tablas una columna denominada \textbf{id} en la primera posición. Esta columna es especial ya que se carga automáticamente. Es \textbf{autonumérica}. Esto garantiza que, para todas las tablas, cada fila tenga un número de identificación único. Dicho de otro modo, no hay en ninguna tabla de sistema dos filas que contengan celdas exactamente iguales.

(Captura de pantalla)

El \textbf{id} es un campo a tener en cuenta cuando se filtra por filas repetidas. Es decir, si se quisiera saber cuáles de las filas de una tabla contienen celdas iguales, será necesario pues excluir de la búsqueda al campo \textbf{id}.

El sistema brinda para todas las tablas las mismas opciones de manipulación de datos. Es decir que para toda tabla definida por el administrador se podrán realizar tareas de:

(Captura de pantalla)

\begin{itemize}
 \item Búsquedas por índice o de modo avanzado
 \item Alta de datos
 \item Modificación de datos 
 \item Eliminación de datos
\end{itemize}

\subsection{Búsquedas e índice}

La primera utilidad del sistema es la capacidad de realizar búsquedas de todo tipo dentro de una tabla. 
Para ello se crearon dos métodos: uno automático que filtra datos por columna de modo sistemático llamado \textbf{Búsqueda Clasificada}. Y el segundo denominado \textbf{Búsqueda Avanzada} que ofrece un formulario con todos los campos dela tabla pàra buscar de acuerdo con diferentes criterios.

\subsubsection{Búsqueda Clasificada}

El de este método realiza un índice de la tabla --como si se tratase del autofiltro en el Excel-- listando todos los nombres de las columnas y debajo los datos sin repetir de esa misma columna, indicando además cuántas filas de cada columna hay sin repetir. Al hacer click en uno de ellos el sistema realiza lo mismo con la salvedad de que se buscarán todos los datos sin repetir que coincidan con ese primer valor elegido.

(Captura de pantalla)

Asi se tiene una aplicación genérica que servirá para filtrar datos sin necesidad de utilizar el teclado. Otra función que tiene esta aplicación esla de ver datos repetidos e inconsistentes que difieran en algún caracter ya que inevitablemente se muestran todos los datos de la tabla.

Es posible filtrar por celdas vacías de la tabla. 


\subsubsection{Búsqueda Avanzada}

Este formulario realiza un análisis de la tabla y, de acuerdo al tipo de dato definido en la columna, trae el formulario correspondiente. Por ejemplo en caso de que se trate de una columna de fecha o de número entero o decimal, el sistema ofrece un formulario que permite realizar búsquedas del tipo similar o exacto, mayor que, menor que y desde-hasta. Si se trata de valores tipo string (o de texto) ofrece un campo simple.

(Captura de pantalla)

Para todos los casos se puede indicar que la búsqueda sea exacta o similar. Búsqueda similar significa que la porción de texto o de número que estamos ingresando se encuentra en alguna parte del resultado.

\paragraph{Ejemplo} Si buscamos el número 123 en una columna de nuestra tabla, y vemos que se encuentra como valor el número 512378, una búsqueda \textbf{exacta} no arrojará ningún valor, mientras que una búsqueda \textbf{similar} si dado que la cadena buscada \textbf{123} se encuentra en 5\textbf{123}78.

En torno de este punto es importante decir que la \textbf{Búsqueda Avanzada} por defecto utiliza el modo \textbf{similar} de búsqueda.

\subsubsection{Datos repetidos y no repetidos}

Para no producir datos repetidos las distintas bases de datos estan configuradas con lo que se denomina un ID unico el cual garantiza que todas las filas de las tablas sean distintas unas de las otras a pesar de que los datos ingresados puedan ser similares en la mayoria de los campos 


\subsection{Alta de datos}

Una de las funciones mas utilizadas en el sistema IDB es la carga de datos, esta consta de una serie de pasos importantes para la correcta insercion de datos en la base de datos. Cada tabla tiene la posibilidad de agregar nuevas filas de datos ya sea por un curso, una carrera, un libro, una nota, etc. Al loguearnos en la base de datos podemos observar las distintas tablas que la conforman y en todas ellas podemos hacer una carga de datos. 

Si necesitamos dar de alta un dato en una tabla específica, podemos ingresar al link correspondiente de “Alta”, este nos redirijirá a un formulario con los distintos campos opcionales que necesitamos almacenar en la tabla elegida. 
Dentro de cada campo podemos encontrar las restricciones que tienen cada uno al momento de ingresar un dato y un ejemplo. A su vez nos ofrece dos opciones para ingresar un dato. Estas constan de un valor ya agregado anteriormente o un conjunto de valores agregados. Tambien hay que tener en cuenta que varios datos pueden ser completados automaticamente ya que estos pueden pertenecer a datos asociados a otros registros internos. 

(Captura de pantalla)

Luego de la carga de los datos, pulsamos en el boton “Enviar” y este realizara la carga a la tabla dentro de la base de datos del Sistema generando un ID unico para ese registro. Nos mostrara un informe con los registros cargados y se regresara al menu principal ya con la base actualizada.

\subsection{Modificacion de datos}

Al momento de corregir un registro o la necesidad de actualizar un dato el sistema IDB proporciona un servicio simple para llevar a cabo esa tarea. Gracias a los sistemas de busqueda podemos encontrar los registros que necesitamos modificar, y cada registro encontrado contiene una funsion directa de edicion. Luego de las modificaciones necesarias al confirmar los cambios estos se guardan en la tabla correspondiente.
 
\subsection{Eliminación de datos}

Para la eliminación de registros el sistema cuenta con una serie de pasos importantes. El usuario está obligado a recorrerlos con el objetivo de asegurar la información que uno desea manipular. Al momento de realizar este tipo de modificación para eliminar un registro el usuario gracias a los distintos sistemas de búsquedas, tiene que encontrar el registro que desea eliminar. Luego de encontrar el registro, el paso siguiente es guardar ese registro como una búsqueda particular. Esto logra que el usuario pueda chequear los datos a eliminar en el menú principal. 

Atención! (es muy importante chequear los datos). Como mencionamos en el punto \ref{ElementosDelMenuPrincipal}, éste contiene la opción “Eliminación de registros”. Al ingresar en dicha opción, nos muestra el listado de búsquedas particulares que tenemos realizadas. Por lo tanto debe existir el registro previamente guardado. Luego de estos pasos podemos proceder a eliminar el registro deseado.

(Captura de pantalla)

Como la eliminación de un registro es un ejercicio muy delicado, el sistema IDB vuelve a mostrar los datos con los que va a realizar este proceso, asegurando que el usuario tenga un repaso de los estos y pueda confirmar la operación. Aunque el ejercicio sea muy tedioso es necesario que se realice una confirmación del proceso una vez más asegurando que el usuario es consciente de lo que está haciendo. No obstante aunque realicemos los pasos anteriores, el registro no fue eliminado aún de la base de datos. Este queda guardado en la papelera con el objetivo de poder recuperar los datos en cualquier momento.

\section{Informes}
\subsection{Edición de informes}
\subsection{Exportación de datos a excel}
El sistema IDB presenta una facilidad al momento de intentar manejar los datos para poder realizar estadisticas externas, busquedas especiales, contaduria de datos y/o el analisis de informacion. Para poder realizar estas tareas se utiliza la exportacion de los datos para poder integrarlos con los softwares que deseamos utilizar. Esta exportacion se realiza en lo que se denomina un archivo CSV o separado por comas, lo cual nos permite integrarlo en cualquier sistema que entienda este formato.

\subsection{Backup por tabla}


\end{document}
