\documentclass[a4paper,10pt]{article}
\usepackage[utf8]{inputenc}%PAQUETE PARA CORRECCION DE TIPOGRAFIA PARA ESPAÑOL
\usepackage[english,spanish]{babel}%PAQUETE DICCIONARIO ESPAÑOL
\usepackage{amssymb}%PAQUETES COMPLEMENTO PARA TIPOGRAFIA MATEMATICA
\usepackage[pdftex]{graphicx}%PAQUETE PARA LA INCLUSION DE IMAGENES jpg, png Y pdf
\usepackage{pdfpages}
\renewcommand{\rmdefault}{phv} % Arial
\renewcommand{\sfdefault}{phv} % Arial
\author{Instituto Dan Beninson}%AUTOR DEL DOCUMENTO
\date{\today}%FECHA DE EDICION DEL DOCUMENTO
\title{Manual del usuario LectorMySQL}%TITULO DEL DOCUMENTO

\begin{document}
\maketitle
\abstract{El sistema de administración IDB es una máscara que facilita el manejo de tablas mysql. El sistema lee todas las bases de datos instaladas en el sistema y crea automáticamente formularios de alta, modificación, baja, consultas avanzadas, informes y otros. Es un sistema configurable y multiusuario capaz de limitar en lectura/escritura por tabla por usuario. 

Fue programado en PHP y MySQL ejecutado como un servicio web en servidor Apache, --actualmente corriendo bajo linux Debian 9-- frente a la necesidad de dar soporte funcional a las dos sedes CAE y CAC y de centralizar dicha información.

El próposito de este sistema es dar soporte administrativo a todas las áreas del instituto. El sistema cuenta con un núcleo principal llamado LectorMySQL y subsistemas dedicados a cada una de las necesidades. Así el Sistema IDB es de uso cotidiano para el manejo de información de alumnos, recursos, notas, biblioteca y otros.
}

\clearpage

\tableofcontents
\clearpage

\section{Introduccion}
\subsection{¿Qué es y para que sirve?}
El sitema de administacion y Alumnos es una base de datos que surge como una 
necesidad puntual de almacenamiento de información consulta e interacción de 
distintas áreas del instituto. 
Este sistema de gestión permite operaciones básicas de consulta y maneja un 
proceso de actualización, reorganización, añadidudra y borrado de información. 
Es lo que se denomina una base de datos dinámica.


\subsection{¿Qué es una base de datos?}
Se llama base de datos, o también banco de datos, a un conjunto de información 
perteneciente a un mismo contexto, ordenada de modo sistemático para su 
posterior recuperación, análisis y/o transmición.
El manejo de las bases de datos se lleva mediante sistemas de gestión (llamados 
DBMS por sus siglas en inglés: Database Management Systems o Sistemas de Gestión 
de Base de Datos).
En la conformación de una base de datos se pueden seguir diferentes modelos y 
paradigmas, cada uno dotado de características, ventajas y dificultades, 
haciendo énfasis en su estructura organizacional, su jerarquía, su capacidad de 
transmisión o de interrelación, etc. Esto se conoce como modelos de base de 
datos y permite el diseño y la implementación de algoritmos y otros mecanismos 
lógicos de gestión, según sea el caso específico.
\subsection{Tecnología}
\section{Ingreso al sistema}
\subsection{Elementos del menú principal}


\section{Entorno de Usuario}
\subsection{Tabla de usuario}
\subsection{Búsquedas particulares}
\subsection{Cambio de contraseña}
\subsection{Importación de datos}
\subsection{Compartir datos con otros usuarios}


\section{Funciones de Tabla}
\subsection{Búsquedas e índice}
\subsection{Alta de datos}
\subsection{Modificacion de datos}
\subsection{Eliminación de datos}


\section{Informes}
\subsection{Edición de informes}
\subsection{Exportación de datos a excel}
\subsection{Backup por tabla}


\end{document}
