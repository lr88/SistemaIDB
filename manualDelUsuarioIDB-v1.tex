\documentclass[a4paper,10pt]{article}
\usepackage[utf8]{inputenc}%PAQUETE PARA CORRECCION DE TIPOGRAFIA PARA ESPAÑOL
\usepackage[english,spanish]{babel}%PAQUETE DICCIONARIO ESPAÑOL
\usepackage{amssymb}%PAQUETES COMPLEMENTO PARA TIPOGRAFIA MATEMATICA
\usepackage[pdftex]{graphicx}%PAQUETE PARA LA INCLUSION DE IMAGENES jpg, png Y pdf
\usepackage{pdfpages}
\renewcommand{\rmdefault}{phv} % Arial
\renewcommand{\sfdefault}{phv} % Arial
\author{Instituto Dan Beninson}%AUTOR DEL DOCUMENTO
\date{\today}%FECHA DE EDICION DEL DOCUMENTO
\title{Manual del usuario LectorMySQL}%TITULO DEL DOCUMENTO

\begin{document}
\maketitle
\abstract{El sistema de administración IDB es una máscara que facilita el manejo de tablas mysql. El sistema lee todas las bases de datos instaladas en el sistema y crea automáticamente formularios de alta, modificación, baja, consultas avanzadas, informes y otros. Es un sistema configurable y multiusuario capaz de limitar en lectura/escritura por tabla por usuario. 

Fue programado en PHP y MySQL ejecutado como un servicio web en servidor Apache, --actualmente corriendo bajo linux Debian 9-- frente a la necesidad de dar soporte funcional a las dos sedes CAE y CAC y de centralizar dicha información.

El próposito de este sistema es dar soporte administrativo a todas las áreas del instituto. El sistema cuenta con un núcleo principal llamado LectorMySQL y subsistemas dedicados a cada una de las necesidades. Así el Sistema IDB es de uso cotidiano para el manejo de información de alumnos, recursos, notas, biblioteca y otros.
}

\clearpage

\tableofcontents
\clearpage

\section{Introduccion}

\subsection{¿Qué es el LectorMySQL y para que sirve?}

Es una aplicación web, multiusuario que permite la gestión de tablas en una mase de datos MySQL. 
Más precisamente, es una interfaz entre el usuario y la información guardada en una base de datos.

Este sistema de gestión permite operaciones básicas de consulta y maneja un 
proceso de actualización, reorganización, alta y baja de información en forma de tablas. 

Genera automáticamente formularios para interactuar con todo tipo de tablas, optimizando el uso y unificando 
el tipo de información. Permite asimismo la gestión compartida de la información ya que el acceso es 
restringido por un usuario Administrador que es quien define los perfiles.

\subsection{¿Qué es una base de datos?}
Se llama base de datos ---o también banco de datos--- a un conjunto de información 
perteneciente a un mismo contexto, ordenada en tablas de modo sistemático para su 
recuperación, análisis y/o transmisión.

El manejo de las bases de datos se lleva mediante sistemas de gestión (llamados 
DBMS por sus siglas en inglés: Database Management Systems o Sistemas de Gestión 
de Base de Datos). MySQL es uno de ellos. Es un sistema de gestión de información que opera 
sobre un servidor web y que queda a la espera de una petición por parte del usuario.

En la conformación de una base de datos se pueden seguir diferentes modelos y 
paradigmas, cada uno dotado de características, ventajas y dificultades, 
haciendo énfasis en su estructura organizacional, su jerarquía, su capacidad de 
transmisión o de interrelación, etc. Esto se conoce como modelos de base de 
datos y permite el diseño y la implementación de algoritmos y otros mecanismos 
lógicos de gestión, según sea el caso específico.

El LectorMySQL es una máscara o interfaz que se conecta a un servidor de base de datos MySQL ---por medio de un servidor web---
 e interactúa con ellas sin importar el tipo o formato de tablas existentes.

\subsection{Tecnología}

\section{Ingreso al sistema}
Para ingresar al Sistema IDB necesitamos previamente un Navegador web (Chrome, Firefox, Opera, etc.). Este sistema al tener una interface web, el usuario ingresa a través de la dirección física o IP del Sistema IDB en la barra de dirección de su navegador preferido. Luego del direccionamiento nos encontramos con una serie de menús. El primero corresponde a “Bases de Datos Activas” en el cual podemos realizar consultas y ABM de todas las bases de datos como por ejemplo Alumnos IDB, Biblioteca IDB, Encuestas IDB, etc.). El segundo menú corresponde a los “Sistemas Dedicados” que complementan al uso del sistema por ejemplo, Contabilidad y Recursos, Sistema de Compras, Calendario, Sistema de Inasistencias.
Para el primer grupo de opciones al querer ingresar nos solicita una autenticación (Usuario y Contraseña), permitiendo o negando el acceso a las diferentes bases de datos. 
En el segundo grupo nos re-direcciona a un sub-menú referente al año el cual queremos gestionar y luego solicita la autenticación del usuario. 
 
\subsection{Elementos del menú principal}
\section{Entorno de Usuario}
\subsection{Tabla de usuario}
\subsection{Búsquedas particulares}
\subsection{Cambio de contraseña}
\subsection{Importación de datos}
\subsection{Compartir datos con otros usuarios}


\section{Funciones de Tabla}

\subsection{Estructura de una tabla de base de datos}

Cuando el administrador da de alta una tabla en una base dados MySQL, es necesario definir una serie de parámetros y características de cada columna o \textbf{campo}. Es decir, que toda tabla del sistema tiene  definidos en detalle el tipo de datos para cada columna, para cada tabla. Por ejemplo definir si el tipo es numérico, de fecha, texto, etc. Esto es necesario para poder realizar una validación\footnote{Se denomina validación al control que se realiza sobre los datos antes de que ingresen en una tabla.} de los datos y garantizar asi la homogeneidad del contenido de la columna y por ende de la tabla.

Por defecto el sistema agrega a todas las tablas una columna denominada \textbf{id} en la primera posición. Esta columna es especial ya que se carga automáticamente. Es \textbf{autonumérica}. Esto garantiza que, para todas las tablas, cada fila tenga un número de identificación único. Dicho de otro modo, no hay en ninguna tabla de sistema dos filas que contengan celdas exactamente iguales.

El \textbf{id} es un campo a tener en cuenta cuando se filtra por filas repetidas. Es decir, si se quisiera saber cuáles de las filas de una tabla contienen celdas iguales, será necesario pues excluir de la búsqueda al campo \textbf{id}.

El sistema brinda para todas las tablas las mismas opciones de manipulación de datos. Es decir que para toda tabla definida por el administrador se podrán realizar tareas de:

\begin{itemize}
 \item Búsquedas por índice o de modo avanzado
 \item Alta de datos
 \item Modificación de datos 
 \item Eliminación de datos
\end{itemize}


\subsection{Búsquedas e índice}

La primera utilidad del sistema es la capacidad de realizar búsquedas de todo tipo dentro de una tabla. Para ello se crearon dos métodos: uno automático que filtra datos por columna de modo sistemático llamado \textbf{Búsqueda Clasificada}. Y el segundo denominado \textbf{Búsqueda Avanzada} que ofrece un formulario con todos los campos dela tabla pàra buscar de acuerdo con diferentes criterios.

\subsubsection{Búsqueda Clasificada}

El algoritmo de este método realiza un índice de la tabla --como si se tratase del autofiltro en el Excel-- listando todos los nombres de las columnas y debajo los datos sin repetir de esa misma columna, indicando además cuántas filas de cada columna hay sin repetir. Al hacer click en uno de ellos el sistema realiza lo mismo con la salvedad de que se buscarán todos los datos sin repetir que coincidan con ese primer valor elegido.

Asi se tiene una aplicación genérica que servirá para filtrar datos sin necesidad de utilizar el teclado. Otra función que tiene esta aplicación esla de ver datos repetidos e inconsistentes que difieran en algún caracter ya que inevitablemente se muestran todos los datos de la tabla.

Es posible filtrar por celdas vacías de la tabla. 


\subsubsection{Búsqueda Avanzada}

Este formulario realiza un análisis de la tabla y, de acuerdo al tipo de dato definido en la columna, trae el formulario correspondiente. Por ejemplo en caso de que se trate de una columna de fecha o de número entero o decimal, el sistema ofrece un formulario que permite realizar búsquedas del tipo similar o exacto, mayor que, menor que y desde-hasta. Si se trata de valores tipo string (o de texto) ofrece un campo simple.

Para todos los casos se puede indicar que la búsqueda sea exacta o similar. Búsqueda similar significa que la porción de texto o de número que estamos ingresando se encuentra en alguna parte del resultado.

\paragraph{Ejemplo} Si buscamos el número 123 en una columna de nuestra tabla, y vemos que se encuentra como valor el número 512378, una búsqueda \textbf{exacta} no arrojará ningún valor, mientras que una búsqueda \textbf{similar} si dado que la cadena buscada \textbf{123} se encuentra en 5\textbf{123}78.

En torno de este punto es importante decir que la \textbf{Búsqueda Avanzada} por defecto utiliza el modo \textbf{similar} de búsqueda.

\subsubsection{Datos repetidos y no repetidos}



\subsection{Alta de datos}
\subsection{Modificacion de datos}
\subsection{Eliminación de datos}


\section{Informes}
\subsection{Edición de informes}
\subsection{Exportación de datos a excel}
\subsection{Backup por tabla}


\end{document}
